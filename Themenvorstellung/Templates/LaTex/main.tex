\input{src/header}
\usepackage{enumitem}
\usepackage{cite}
    \bibliographystyle{IEEEtran}
\usepackage{natbib}
\newcommand{\dozent}{Dr. Larissa Groth}
\newcommand{\veranstaltung}{IoT Network Security}
\newcommand{\semester}{SoSe23}
\newcommand{\studenten}{Aiman Al-Hazmi, Zohreh Asadi}

\begin{document}
% /////////////////////// BEGIN TITLEPAGE /////////////////////////
\begin{titlepage}
	\title{\veranstaltung}
	\subtitle{\Large Untersuchung der verschieden Schutzmechanismen in Smart Home Netzwerken, \semester}
	\author{\textbf{Autoren:} \studenten \\ \textbf{Dozentin:} \dozent}
	\date{\normalsize \today}
\end{titlepage}

\maketitle								% Erstellt das Titelblatt
\vspace*{-9cm}							% rückt Logo an den oberen Seitenrand
\makebox[\dimexpr\textwidth+1cm][r]{	%rechtsbündig und geht rechts 1cm über Layout hinaus
	\includegraphics[width=0.4\textwidth]{src/fu_logo} % fügt FU-Logo ein
}
% /////////////////////// END TITLEPAGE /////////////////////////

\vspace{6cm}							% Abstand
\rule{\linewidth}{0.8pt}				% horizontale Linie
\tableofcontents
\newpage

\section{Einführung}
\subsection{Hintergrund und Bedeutung der IoT-Netzwerksicherheit in Smart Homes
}
Smart Homes sind zunehmend verbreitet, da das Internet der Dinge (IoT) weiter wächst. Die weite Verbreitung dieser Technologien hat jedoch Bedenken hinsichtlich ihrer Sicherheit und Privatsphäre aufgeworfen. Um diese Bedenken anzugehen, haben Forscher zahlreiche Studien durchgeführt, um die Sicherheitsrisiken und Schutzmaßnahmen in Smart Home-Umgebungen zu identifizieren und zu bewerten.

In diesem Aufsatz wird ein Überblick über die Sicherheits- und Datenschutzprobleme in Smart Home-Umgebungen gegeben, wobei der Schwerpunkt auf der Untersuchung der verschiedenen Schutzmechanismen in Smart Home-Netzwerken liegt. Unsere Übersicht stützt sich auf eine Auswahl wissenschaftlicher Artikel, darunter die Arbeiten von (...), sowie Bücher wie (...) . 
\subsection{Zielsetzung und Forschungsfragen}
Ziel dieser Arbeit ist es, eine detaillierte Zusammenfassung der aktuellen Forschungsergebnisse zu geben und Empfehlungen für den Einsatz von Schutzmechanismen in Smart Home-Umgebungen zu geben.

\newpage
\section{IoT-Netzwerksicherheit in Smart Homes}

\subsection{Architektur von Smart Home-Netzwerken}
\subsection{Bedrohungen und Risiken für Smart Home-Netzwerke}
\subsection{Wichtige Schutzmechanismen zur Sicherung von Smart Home-Netzwerken}

\newpage
\section{Verschlüsselung in Smart Home-Netzwerken}

\subsection{Symmetrische und asymmetrische Verschlüsselung}
\subsection{Verschlüsselung von Datenübertragungen und Speichermedien in Smart Home-Netzwerken}
\subsection{Vor- und Nachteile der Verschlüsselungstechnologien}

\newpage
\section{Authentifizierung und Zugriffskontrolle in Smart Home-Netzwerken}

\subsection{Benutzer- und Geräte-Authentifizierung}
\subsection{Zugriffskontrolle und Berechtigungen}
\subsection{Multifaktor-Authentifizierung in Smart Home-Netzwerken}

\newpage
\section{Best Practices und Implementierungsbeispiele}

\subsection{Best Practices für die IoT-Netzwerksicherheit in Smart Homes}
\subsection{Herausforderungen und Einschränkungen der Implementierung von Schutzmechanismen in Smart Home-Netzwerken}

\newpage
\section{Zusammenfassung und Ausblick}

\subsection{Zusammenfassung der Ergebnisse}
\subsection{Ausblick auf zukünftige Entwicklungen und Forschungsbedarf}

\newpage

\section{Zusammenfassung der besten Quellen}
\subsection{Quelle: \cite{khatoun2022cybersecurity}} 
    
\newpage

    \nocite{*}
    \bibliography{src/referencias}


\end{document}
